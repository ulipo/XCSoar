\chapter{Introduction}\label{cha:introduction}
Questo è il manuale del pilota di XCSoar, un computer di volo open-source
originalmente sviluppato per i dispositivi Pocket PC devices.  Si presume 
che l'utente abbia un solida conoscenza della teoria del volo veleggiato,
e come minimo una conoscenza basilare del volo di cross-country.

Gli aggiornamenti al software di XCSoar potrebbero rendere questo manuale 
obsoleto. Si raccomanda di leggere le note di rilascio distribuite con il 
software per tenere tracia dei cambiamenti.  Aggionamenti a manuale e software
sono disponibili su 
\begin{quote}
\xcsoarwebsite{}
\end{quote}

\section{Organizazione di questo manuale}

\todonum[inline]{Write about the manual crossref hinting icons and the yellow
colour. The Quickstart will be readable also without those links available} 
Questo manuale è scritto particolarmente per permettere all'utente di XCSoar
un rapido utilizzo, 
  \emph{così come} supportare una profonda comprensione di tutte le funzioni, 
concetti e tattiche introdotte. Gli autori hanno fatto tutto il possibile per 
farlo dalla prospettiva del pilota (e onestamente sperano di esserci riusciti).

Gli autori suggeriscono vivamente di prendersi il tempo di leggere a fondo
tutto il manuale, capitolo per capitolo (con l'eccezione dei capitoli di 
riferimento Infobox e Configurazione). È garantito che il tempo speso si
ripagherà sotto forma di comprensione. Durante la lettura potrà capitare di 
sentirsi un po' persi ogni tanto, per questo motivo gli autori hanno inserito
link ed icone.

\begin{figure}[h]
\centering
\includegraphics[width=0.8cm,angle=0,keepaspectratio='true']{figures/config.pdf}
\hspace{1.5cm}
\includegraphics[width=0.8cm,angle=0,keepaspectratio='true']{figures/reminder.pdf}
\hspace{1.5cm}
\includegraphics[width=0.8cm,angle=0,keepaspectratio='true']{figures/gesture.pdf}
\hspace{1.5cm}
\includegraphics[width=0.8cm,angle=0,keepaspectratio='true']{figures/warning.pdf}
\caption{Icons configuration, reminder, gesture, warning}
\end{figure}

\warning Avvertimento. L'icona di avvertimento è usata ovunque le cose vadano 
seguite esattamente. Non farlo potrebbe causare risultati inattesi, disfunzioni
totali o addirittura pericolo per l'incolumità. Procedere solo se si è compreso
il pericolo.

\gesture{DU} Gesto. Nei dispositivi dotati di touch-screen, un gesto di scorrimento
può essere usato per invocare il menu o una funzione piuttosto che un'altra.
In questo esempio, DU sta per muovere il dito verso il basso e poi verso l'alto
(in linee diritte) sullo schermo.
  
\gesturespec{du} Gesto Specifico. Ovunque l'auore abbia tentuto il passo con il rapido sviluppo di XCSoar, è fornita un'icona specifica che illustri i movimenti.

\tip Promemoria. Questa icona rappresenta un suggerimenti, un trucco, cose che dovresti ricordare dopo aver letto la sezione corrispondente e così via.

\config{orientation} Vedi configurazione ... L'icona che rappresenta du e attrezzi
da officina rimanda ad un approfondimento degli elementi menzionati e come configurarli. I numeri accanto all'icona si riferiscono agli specifici capitoli/sezioni di riferimento di questo manuale \ref{cha:infobox} e 
\ref{cha:configuration}, in questo caso si riferiscono ad una sezione \ref{conf:orientation}. 

\marginlabel{\parbox{1.3cm}{\rotatebox[origin=c]{180}{\includegraphics[width=0.9cm]{figures/warning.pdf}}}}
\rotatebox[origin=c]{180}{Smetti di leggere il manuale mentre stai volando all'indietro!}

\emph{Leggi} a casa, \emph{configura} al suolo, in sicurezza. Se hai percepito questo avvertimento (invertito) 
come un esempio, sei pronto a procedere.

\config{usingxcsoarsafely} In riferimento alla seconda icona di esempio 
``configurazione'' a sinistra, l'icona punta al capitolo \ref{cha:introduction}, (questo capitolo), sezione 
\ref{sec:usingxcsoarsafely}, ``Usare XCSoar in sicurezza'' sotto, che potrebbe essere
intesa come ``come configurare te stesso''. Sta a te decidere se tuffarti in una discussione
approfondita e poi tornare indietro, oppure semplicemente procedere. Se stai leggendo il manuale in 
forma elettronica, cliccare il numero ti farà saltare al riferimento incrociato richiesto.
Usa la funzione ``indietro'' (o simili) del tuo browser per tornare sul capitolo da cui eri saltato.

I numeri stampati in blu, così come le icone appena introdotte, indicano ``aiuto
disponibile''. Lo stesso vale per gli altri URL e testi blu sottostanti.
Cliccare su un testo come \xcsoarwebsite{/contact} aprità il tuo browser web 
o l'applicazione di posta elettronica per metterti in contatto rispettivamente
con altre risorse o gente più esperta.

Il resto di questo capitolo ``Introduzione'' è riguardo a renderti preparato
per XCSoar, come aumentare il tuo livello di comprensioe e coltivare  le tue capacità. 
Il capitolo \ref{cha:quickstart} ``Quickstart'' potrebbe essere la prossima boa dopo
\ref{cha:installation} ``Installazione'' per gli utenti frettolosi. Sentiti libero di 
tagliar corto, ma non disdegnare di leggere tutto capitolo per capitolo, seguendo:

Il capitolo~\ref{cha:interface} introduce i concetti dell'interfaccia utente
e fa una panoramica del display.

Il capitolo~\ref{cha:navigation} descrive in gran dettaglio la mappa mobile
e come il software possa assistere nella navigazione generale.

Il capitolo~\ref{cha:tasks} descrive come sono specificate e volate le task
di cross-country, e presenta alcuni degli strumenti di analisi disponibili
pe ri piloti per aiutarli a migliorare le proprie prestazioni.

Il capitolo~\ref{cha:glide} scende in maggior dettaglio nelle funzioni del computer
di volo, perchè è importante per i piloti essere a conoscenza di come questo 
esegua i suoi calcoli.

Il capitolo~\ref{cha:atmosph} spiega come il computer si possa
interfacciare con variometri ed altri sensori di dati dell'aria
e come usa questi dati per fornire vari modelli dell'atmosfera, in
particolare su vento e convezione termica.

Il capitolo~\ref{cha:airspace} descrive come XCSoar può assistere 
nella gestione del volo in determinati airspace ed il sistema
anticollisione FLARM.

Il capitolo~\ref{cha:avionics-airframe} riguarda integrazione
 di sistema e diagnostiche, l'integrazione di XCSoar con
dispositivi di comunicazione e interruttori del velivolo.

Il resto del manuale contiene principalmente materiale di riferimento.
Il capitolo~\ref{cha:infobox} elenca i tipi di informazione che possono
essere mostrati nella griglia di InfoBox accanto alla mappa.
La configurazione del software è descritta in dettaglio nel capitolo~\ref{cha:configuration}.
I formati dei faile dati usati dal programma, come ottenerli
e come modificarli, è descritto nel capitolo~\ref{cha:data-files}.

Infine, una breve storia e discussione del processo di sviluppo di XCSoar
è presentata nel capitolo~\ref{cha:history-development}.

\section{Note}

\subsection*{Screenshot}
In tutto il manuale ci sono diversi screenshot di XCSoar. Questi sono presi
dal programma in funzione su una varietà di piattaforme hardware e probabilmente
anche versioni differenti. Ogni piattaforma potrebbe avere differente risoluzione
dello schermo, layout e caratteri, quindi ci possono essere leggere differenze 
nell'aspetto del display. La maggior parte degli screenshot in questo manuale
sono presi con XCSoar in configurazione orizzontale.

\section{Piattaforme}
\begin{description}
\item[Dispositivi Android]
XCSoar funziona su Android 1.6 o successivi.
\item [eBookreader]
XCSoar funziona su diversi modelli di eReader Kobo.
\item[PC Windows]
È possibile far funzionare XCSoar su un qualsiasi computer con sistema
operativo Windows (Vista e successivi). 
Questo setup può essere usato per allenarsi ad usare XCSoar.
Nel software sono presenti una modalità simulazione ed una funzione
di replay delle tracce IGC che possono essere usate in assenza di un
dispositivo GPS valido.
\item[Unix/Linux PC]
XCSoar funziona su Unix/Linux.
\end{description}



\section{Supporto tecnico}

\subsection*{Troubleshooting}
XCSoar è prodotto da un piccolo team di sviluppatori dedicati. Sebbene siamo
felici di aiutare con l'utilizzo del software, non possiamo insegnarti le basi 
della moderna tecnologia informatica. Se hai una particolare domanda su XCSoar 
non presente in questo manuale, per favore mettiti in contatto.
Troverai tutti i seguenti link riassunti su:
\begin{quote}
\xcsoarwebsite{/contact}
\end{quote}
Per cominciare con la comunicazione, unisciti al forum di XCSoar:
\begin{quote}
\url{http://forum.xcsoar.org}
\end{quote}
Se i tuoi dubbi non sono già stati affrontati, postali o scrivici: 
\begin{quote}
\href{mailto:xcsoar-user@lists.sourceforge.net}{xcsoar-user@lists.sourceforge.net}
\end{quote}
Le domande frequenti verranno aggiunte a questo documento.
Potresti anche trovare utile iscriverti alla mailing-list degli utenti di XCSoar,
in modo di essere sempre aggiornato sugli ultimi sviluppi.

Se niente di tutto questo è sufficiente, probabilmente hai scoperto un bug.

\subsection*{Feedback}
Come ogni programma software complesso, XCSoar può essere soggetto a
bug, quindi se ne trovi qualcuno per favore riportalo agli sviluppatori
usando il nostro portale di bug tracking: 
\begin{quote}
\xcsoarwebsite{/develop/new_ticket.html}
\end{quote}
o inviando una mail a:
\begin{quote}
\href{mailto:xcsoar-devel@lists.sourceforge.net}{xcsoar-devel@lists.sourceforge.net}
\end{quote}
XCSoar registra molti dati valevoli in un file di log
\verb|xcsoar.log| nella directory \texttt{XCSoarData}. Il file di log può essere aggiunto alla segnalazione del bug per aiutare gli sviluppatori a determinare la causa del problema.
Se ti piace l'idea di fare qualcosa di più, fatti coinvolgere:
\begin{quote}
\xcsoarwebsite{/develop}
\end{quote}

\subsection*{Aggiornamenti}
È consigliato visitare periodicamente il sito di XCSoar per controllare
se ci sono aggiornamenti del programma.
La procedura di installazione descritta nel capitolo seguente può essere 
ripetuta per aggiornare il software.
Tutte le configurazioni utente e file di dati saranno preservate durante
l'aggiornamento o la reinstallazione.

È anche consigliato di controllare periodicamente se ci sono aggiornamenti
per i file dei dati, specialmente quelli degli spazi aerei, che potrebbero
essere soggetti a cambiamenti dalle autorità di aviazione civile nazionali.

\section{Allenamento}
Per la sicurezza propria e degli altri, ai piloti è consigliato di allenarsi
a terra all'uso di XCSoar, in modo da familiarizzare con l'interfaccia e le
funzioni prima di volare.

\subsection*{Usare XCSoar sul PC}
La versione per PC può essere usata per familiarizzare con l'interfaccia
e le funzionalità di XCSoar nel comfort della propria casa. Tutti i file e
le configurazioni usate da questa versione sono identici alle versioni embedded,
quindi può essere utile provare le personalizzazioni al PC prima di usarle
in volo.

La versione da PC può anche esser collegata a dispositivi esterni e funzionare
esattamente come la versione embedded. Utilizzi suggeriti sono:
\begin{itemize}
\item Connettere il PC ad un dispositivo FLARM per usare XCSoar come una
stazione a terra, per visualizzare il traffico FLARM.
\item Connettere il PC ad un variometro intelligente, come il Vega, per provare
le impostazioni del variometro.
\end{itemize}

\subsection*{Usare XCSoar con un simulatore di volo}
Un buon modo di imparare a dusare XCSoar è di connettere uno smartphone
ad un PC con un simulatore di volo che possa produrre sentenze NMEA
attraverso la porta seriale.
Simulatori di volo adatti sono Condor e X-Plane.  

I benefici di questa forma di allenamento sono che XCSoar può essere usato
in modalità FLY, così si comporta esattamente come se si stesse realmente
volando, e si può ottenere una buona impressione di come il programma lavori
mentre si vola nel simulatore.

\section{Usare XCSoar in sicurezza}\label{sec:usingxcsoarsafely}\label{conf:usingxcsoarsafely}
L'uso di un sistema interattivo come XCSoar in volo comporta un certo
rischio dovuto alla potenziale distrazione del pilota dal mantenimento
della consapevolezza della situazione e dal tenere d'occhio i dintorni.

La filosofia dietro design e sviluppo del software è di tentare di ridurre
il più possibile queste distrazioni minimizzando la necessità di interazione
da parte dell'utente, e presentando le informazioni in modo chiaro ed
interpretabile con uno sguardo.

I pilot che usano XCSoar devono prendersi la responsabilità di usare il
sistema in sicurezza.
Buone pratiche nell'uso di XCSoar includono:
\begin{itemize}
\item Familiarizzare col sistema attraverso intenso allenamento a terra.
\item Effettuare manovre di allontanamento prima di interagire con XCSoar
  in volo, per assicurarsi che non ci sia rischio di collisione con altro traffico.
\item Impostare il sistema in modo da trarre vantaggio dalle funzioni automatiche
  e dagli eventi in ingresso, in modo che le interazioni utente siano mimimizzatee.
  Se ti accorgi di ripetere meccanicamente certe azioni, chiediti (o chied ad altri
  utenti) se il software non possa gestire queste interazioni per te.
\end{itemize}
